%----------------------------------------------------------------------------------------
%	PACKAGES AND OTHER DOCUMENT CONFIGURATIONS
%----------------------------------------------------------------------------------------

\documentclass[10pt]{developercv} % Default font size, values from 8-12pt are recommended

%----------------------------------------------------------------------------------------

\begin{document}

%----------------------------------------------------------------------------------------
%	TITLE, CONTACT INFORMATION AND PICTURE
%----------------------------------------------------------------------------------------

\begin{minipage}[t]{0.47\textwidth} % 45% of the page width for name
	\vspace{-\baselineskip} % Required for vertically aligning minipages
	\vspace{12pt}

	% If your name is very short, use just one of the lines below
	% If your name is very long, reduce the font size or make the minipage wider and reduce the others proportionately
	\colorbox{black}{{\HUGE\textcolor{white}{\textbf{\MakeUppercase{Antoine}}}}} % First name

	\colorbox{black}{{\HUGE\textcolor{white}{\textbf{\MakeUppercase{Barthelemy}}}}} % Last name

	\vspace{6pt}

	{\huge Platform Engineer} % Career or current job title
\end{minipage}
\begin{minipage}[t]{0.28\textwidth} % 27.5% of the page width for the first row of icons
	\vspace{-\baselineskip} % Required for vertically aligning minipages
	\vspace{9pt}
	% The first parameter is the FontAwesome icon name, the second is the box size and the third is the text
	% Other icons can be found by referring to fontawesome.pdf (supplied with the template) and using the word after \fa in the command for the icon you want
	\icon{MapMarker}{12}{Rennes, France}\\
	\icon{Github}{12}{\href{https://github.com/Ant0wan}{github.com/Ant0wan}}\\
	\icon{Phone}{12}{+33 6 95 45 06 11}\\
	\icon{At}{12}{\href{mailto:antoine@linux.com}{antoine@linux.com}}\\
\end{minipage}
\begin{minipage}[t]{0.275\textwidth} % 27.5% of the page width for the second row of icons
	\vspace{-\baselineskip} % Required for vertically aligning minipages
	% The first parameter is the FontAwesome icon name, the second is the box size and the third is the text
	% Other icons can be found by referring to fontawesome.pdf (supplied with the template) and using the word after \fa in the command for the icon you want
	%\hspace{0.5cm}
	\roundpic[xshift=0cm,yshift=0cm]{3.7cm}{3.7cm}{profil.jpg}
\end{minipage}

\vspace{0.5cm}

%----------------------------------------------------------------------------------------
%	INTRODUCTION
%----------------------------------------------------------------------------------------

\cvsect{Who Am I?}

\hspace{6pt}\begin{minipage}[t]{0.98\textwidth} % 40% of the page width for the introduction text
	\vspace{-\baselineskip} % Required for vertically aligning minipages

	{Hello, I am a Platform Engineer with +3 years of experience in multi-cloud deployments, infrastructure automation, container orchestration, and CI/CD. I have expertise in technologies such as AWS, GCP, Kubernetes, Terraform, Ansible, Python and Rust.} % Dummy text

\end{minipage}
\hfill % Whitespace between

%----------------------------------------------------------------------------------------
%	FAVORITE SKILLS AND TECHNOLOGIES
%----------------------------------------------------------------------------------------


\cvsect{My Favorite Stack}
\begin{center}
	\colorbox{black}{{\textcolor{white}{\textbf{\MakeUppercase{Rust}}}}}
	\colorbox{black}{{\textcolor{white}{\textbf{\MakeUppercase{Python}}}}}
	\colorbox{black}{{\textcolor{white}{\textbf{\MakeUppercase{Kubernetes}}}}}
	\colorbox{black}{{\textcolor{white}{\textbf{\MakeUppercase{Terraform}}}}}
	\colorbox{black}{{\textcolor{white}{\textbf{\MakeUppercase{Ansible}}}}}
	\colorbox{black}{{\textcolor{white}{\textbf{\MakeUppercase{AWS}}}}}
	\colorbox{black}{{\textcolor{white}{\textbf{\MakeUppercase{Linux}}}}}
\end{center}

%----------------------------------------------------------------------------------------
%	EXPERIENCE
%----------------------------------------------------------------------------------------

\cvsect{Experience}

\begin{entrylist}
	\entry
		{05/2022 --\\07/2023\\\footnotesize{Full Time}}
		{Platform Engineer}
		{Contentsquare - Rennes, France}
		{Helped in maintaining infratructure as code focusing on various DevOps aspects and responsibilities.\\
		Contributed to the development and implementation of technologies such as Kubernetes, Ansible, Terraform, Spinnaker, Github Actions.\\
		Successfully managed multi-cloud deployments across Azure and AWS, optimizing cloud resources and maintaining high availability.\\
		Supported developers in delivering new features, services, and software releases while troubleshooting issues to ensure seamless operations.\\
		Demonstrated coding skills in Shell, Python, Go and Rust to automate processes and enhance efficiency.\\
		\texttt{AWS}\slashsep\texttt{Azure}\slashsep\texttt{Go}\slashsep\texttt{Github Actions}\slashsep\texttt{Python}\slashsep\texttt{Jenkins}\slashsep\texttt{Terraform}\slashsep\texttt{Ansible}\slashsep\texttt{Spinnaker}\slashsep\texttt{Kubernetes}\slashsep\texttt{Vault}\slashsep\texttt{Helm}}
	\entry
		{06/2021 --\\02/2022\\\footnotesize{Full Time}}
		{DevOps Engineer}
		{SII - Rennes, France}
		{Maintained 4GKE of \textasciitilde5nodes with \textasciitilde300 simultaneous users and their CI/CD for Brittany-Ferries.\\
Migrated 4 terraformed GKE to 4 new GKE clusters with Fluxcd, ConfigConnector, Istio, Google KMS and Sops.\\
		\texttt{Kubernetes}\slashsep\texttt{GCP}\slashsep\texttt{Terraform}\slashsep\texttt{GitlabCI}\slashsep\texttt{FluxCD}}
	\entry
		{11/2020 --\\05/2021\\\footnotesize{Fixed-term}}
		{DevOps Engineer}
		{CIL - Lamballe, France}
		{Deployed a Kubernetes cluster on bare metal servers and migrated applications.\\
		\texttt{Kubernetes}\slashsep\texttt{Docker}\slashsep\texttt{Ansible}\slashsep\texttt{VMWare}\slashsep\texttt{Linux}\slashsep\texttt{Bash}}
	\entry
		{7/2020 --\\11/2020\\\footnotesize{Internship}}
		{DevOps Intern}
		{CIL - Lamballe, France}
		{Gathered infrastructure team requierements and developed a deamon preventing ransomware and users from damaging network attached storages.\\
Deployed 3CX VoIP and maintained all Linux virtual machines and their services.\\
		\texttt{Rust}\slashsep\texttt{Linux}\slashsep\texttt{VMWare}\slashsep\texttt{SQL}\slashsep\texttt{Bash}}
\end{entrylist}

%----------------------------------------------------------------------------------------
%	CERTIFICATIONS
%----------------------------------------------------------------------------------------

\cvsect{Certifications}

\begin{entrylist}
	\certentry
		{11/2023}
		{Linux Foundation Certified System Administrator (LFCS)}
	\certentry
		{08/2023}
		{DevOps for Network Engineers (LFS266)}
	\certentry
		{08/2023}
		{DevOps and SRE Fundamentals: Implementing Continuous Delivery (LFS261)}
	\certentry
		{05/2023}
		{HashiCorp Certified: Terraform Associate (003)}
	\certentry
		{04/2022}
		{Introduction to GitOps (LFS169)}
	\certentry
		{04/2022}
		{Containers Fundamentals (LFS253)}
\end{entrylist}

%----------------------------------------------------------------------------------------
%	EDUCATION
%----------------------------------------------------------------------------------------

\cvsect{Education}

\begin{entrylist}
	\entry
		{2018 - 2023}
		{42}
		{Paris, France}
		{Certificate of Architect in Digital Technologies}
	\entry
		{2014 - 2017}
		{ESSEC}
		{Cergy, France}
		{Bachelor of Business Administration}
\end{entrylist}

%----------------------------------------------------------------------------------------
%	ADDITIONAL INFORMATION
%----------------------------------------------------------------------------------------

\begin{minipage}[t]{0.5\textwidth}
	\vspace{-\baselineskip} % Required for vertically aligning minipages

	\cvsect{Languages}

	\textbf{French} - native\\
	\textbf{English} - proficient
\end{minipage}
\hfill
\begin{minipage}[t]{0.5\textwidth}
	\vspace{-\baselineskip} % Required for vertically aligning minipages

	\cvsect{Other stack/tools used}

	{C, Git, ArgoCD, Latex, Graphviz, Jira}

\end{minipage}

%----------------------------------------------------------------------------------------

\end{document}
